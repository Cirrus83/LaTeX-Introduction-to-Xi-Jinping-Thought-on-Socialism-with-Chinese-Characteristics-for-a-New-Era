% 习概课程论文 LaTeX高仿Word o(^▽^)o 20251119
% 注意:由于ctexart的特性,中文段落内的空格需要用"\ "(强制空格)键入
% PKUTeX会使用Fandol系列字体,可能与Word默认的字体有些许不同
\documentclass[UTF8,12pt,a4paper]{ctexart}
\usepackage[top=2.5cm, bottom=2.5cm, left=3.2cm, right=3.2cm]{geometry}
\usepackage{graphicx}
\usepackage{stackengine}
\usepackage{anyfontsize}
\usepackage{footmisc}
\usepackage{setspace}
\usepackage{xeCJK}
\onehalfspacing  % 全文1.5倍行距

\ctexset{% 不想写注释
    section = {
        format = \zihao{4}\heiti\raggedright\bfseries,
        name = {},
        number = \chinese{section}、,
        aftername = \hspace{0.5em},
        beforeskip = 3.5ex plus 1ex minus .2ex,
        afterskip = 2.3ex plus .2ex
    },
    subsection = {
        format = \zihao{4}\heiti\raggedright\selectfont, % 去掉\bfseries就是不加粗
        name = {},
        number = (\chinese{subsection}), % 中文数字加括号
        aftername = \hspace{0.5em},
        beforeskip = 2.5ex plus 1ex minus .2ex,
        afterskip = 1.8ex plus .2ex
    }
}
% 若本地编译,且使用macOS,以下内容请去除注释,否则文档会默认使用华文系列字体(该部分对于Windows尚未调试)
% \setCJKmainfont{Songti SC}[
%     BoldFont = * Bold,  
%     AutoFakeBold = false,
%     AutoFakeSlant = 0.2  % 启用伪斜体,0.2是倾斜角度
% ]
% \newCJKfontfamily{\songtisc}{Songti SC}
% \newCJKfontfamily{\simhei}{SimHei}[
%     AutoFakeBold = 2.0
% ]
% \newCJKfontfamily{\kaitisc}[
%     UprightFont = * Regular,
%     BoldFont = * Bold,
%     AutoFakeBold = false
% ]{Kaiti SC}
% \renewcommand{\songti}{\songtisc}
% \renewcommand{\heiti}{\simhei}
% \renewcommand{\kaishu}{\kaitisc}
\renewcommand{\footnotesize}{\zihao{5}}

\begin{document}
\begin{titlepage}
    \centering
    \vspace{2.5cm}
    
    \includegraphics[width=0.5\textwidth]{logo_horizontal.eps}\\
    \vspace{2.5cm}
    
    {\zihao{-0}\heiti\textbf{习思想概论课}}\\
    \vspace{0.5cm}
    {\zihao{-0}\heiti\textbf{1班课程论文}}\\
    \vspace{2cm}
    {\songti\zihao{2}题目:\stackon{\rule{12cm}{0.6pt}}{\centering\heiti\zihao{3}论文标题}}\\
    \vspace{3cm}
    
    \noindent\centering\begin{minipage}{0.85\textwidth}
        \zihao{-3}\selectfont\heiti
        花名册序号:\stackon{\rule{9cm}{0.6pt}}{\centering\fangsong\zihao{3} 键入} 
        \vspace{15pt}
        
        \:\;姓\qquad 名:\:\;\stackon{\rule{9cm}{0.6pt}}{\centering\fangsong\zihao{3} 键入} 
        \vspace{15pt}
        
        \:\;学\qquad 号:\:\;\stackon{\rule{9cm}{0.6pt}}{\centering\fangsong\zihao{3} 键入} 
        \vspace{15pt}
        
        \:\;院\qquad 系:\:\;\stackon{\rule{9cm}{0.6pt}}{\centering\fangsong\zihao{3} 数学科学学院} 
        \vspace{15pt}
        
        \:\;主管老师:\:\;\stackon{\rule{9cm}{0.6pt}}{\centering\fangsong\zihao{3} 键入} 
        \vspace{15pt}
        
        \:\;助\qquad 教:\:\;\stackon{\rule{9cm}{0.6pt}}{\centering\fangsong\zihao{3} 键入}
    \end{minipage}

    \vspace{2cm}
    \zihao{3}{二〇二五\heiti 年\songti 十一\heiti 月}
    
    \vfill

\end{titlepage}
\noindent\begin{minipage}{1\textwidth}
    \kaishu\zihao{-4}\textbf{\addCJKfontfeatures{AutoFakeBold=2.0}摘要:}这是一段可以自动换行的长文本,因为它不在命令内部,而是在常规的文本环境中。可以自动换行吗?似乎确实可以。\\
    \textbf{\addCJKfontfeatures{AutoFakeBold=2.0}关键词:}嗷嗷;啊啊
\end{minipage}
\section{xx\ 一级标题\ 黑体\ 四号\ 加粗}
\subsection{xx\ 二级标题\ 黑体\ 四号}
正文:宋体小四\ 1.5倍行间距\ 段落首行缩进两字符,测试自动换行呜呜呜呜呜呜呜呜呜呜呜呜呜呜。\par
脚注:宋体五号,例如:\footnote{图书:《习近平谈治国理政》第2卷,外文出版社,2017年版,第112-113页。}
\footnote{网站:习近平:《xxx》,人民网,x年 x月 x日。}
\footnote{报纸:人名:《xxx》,《人民日报》,x年 x月 x日。}
\footnote{期刊:人名:《xxx》,《中国社会科学》,x年第x期,第1-2页。}

% 参考文献(不想写注释)
\vspace{1cm}
\noindent\begin{minipage}{\textwidth}
    \singlespacing
    \begin{center}
    {\zihao{3}\heiti\textbf{参考文献}}
    \end{center}
    \songti\zihao{5}
    [1]《习近平谈治国理政》第2卷,外文出版社,2017年版。% 要确保每行之间空一行

    [2]习近平:《xxx》,人民网,x年 x月 x日。
    
    [3]人名:《xxx》,《人民日报》,x年 x月 x日。

    [4]人名:《xxx》,《中国社会科学》,x年第x期,第x-x页。
\end{minipage}

\end{document}
